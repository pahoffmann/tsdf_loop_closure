\chapter{Einleitung}

Diese Arbeit beschäftigt sich mit der Konzeption und Implementation einer Lösung zum \emph{Loop Closure  (Schleifenschluss)}-Problem in \textit{Truncated Signed Distance Fields} (TSDF) basiertem \textit{Simulatneous Localization and Mapping} (SLAM) und knüpft an den Konzepten von Eisoldt et al. \cite{HATSDF} an, die einen TSDF-basierten, hardwarebeschleunigten SLAM Ansatz (\emph{HATSDF-SLAM}) entwickelt und dessen Effizienz und Funktionalität verifiziert haben.

Unter SLAM versteht man den Prozess der Erstellung einer Karte einer unbekannten Umgebung, bei gleichzeitiger Positionsbestimmung innerhalb der erstellten Karte.
Das SLAM Problem ist ein \textit{Henne-Ei-Problem}, da eine vollständige Karte benötigt wird, um die Pose eines Roboters akkurat bestimmen zu könnnen, während auf der anderen Seite eine akkurate Pose benötigt wird, um eine vollständige Karte aufbauen zu können. Für die Lösung dieses Problems gibt es sowohl in 2-D als auch 3-D verschiedene Lösungsansätze, die in Kapitel \ref{chapter:sdf} grob eruiert werden.
Diese Arbeit beschränkt sich auf 6-D SLAM. Das Schleifenschluss-Problem in der mobilen Robotik bezeichnet die Herausforderung der Identifikation geschlossene Kreise in abgelaufen Pfaden durch die im SLAM-Prozess erschlossene, zunächst unbekannte Umgebung. Nach der Identifikation einer Schleife zwischen zwei Posen $P_i$ und $P_j$ wird die approximative Transformation $T_{i \rightarrow j}$ bestimmt und mit der Transformation $T_{i \rightarrow j}^{graph}$  verglichen, die aus derzeitigen Pose-Graph hervorgeht. Die Differenz $T_{error}$ zwischen diesen Transformationen wird näherungsweise als der Fehler angesehen, der sich bei der Durchführung des vorliegenden SLAM-Ansatzes zwischen den Posen $P_i$ und $P_j$ akkumuliert hat. In einem Optimierungsschritt wird der Fehler $T_{error}$ durch eine Anpassung des Pose-Teilgraphen zwischen $P_i$ und $P_j$ kompensiert. Dies sorgt für eine konsistenten Pfad und damit verbunden für eine akkurate Karte.  Als Basis für die Lösung dieses Problems dient der Ansatz von Lu und Milios \cite{lu1997globally}, die ein Netzwerk aus 2-D-Poserelationen erstellen, das sowohl Relationen enthält, die aus dem Scan-Matching abstammen, als auch Relationen aus Odometrie Messungen \cite{lu1997globally}. 
Borrmann et al. \cite{borrmann2008globally} erweitern den Ansatz von \cite{lu1997globally} auf drei Dimensionen und erstellen einen Pose-Graphen zur Speicherung von Pose-Relationen. Zur Registierung aufeinanderfolgender Datenscans verwenden Borrmann et al. \cite{borrmann2008globally} den \textit{Pairwise Iterative Closest Point Algorithm} (Pairwise ICP). Zur Detektion eines Schleifenschlusses wird eine einfache Distanz-Heuristik verwendet. Shang et al. \cite{shan2020lio} beschreiben eine vergleichbare Heuristik zur Identifikation von Schleifenschlüssen. Der gewählte Ansatz verwende eine naive, aber effektive Heuristik, die auf der euklidischen Distanz zwischen den Posen des Graphen basiert. Posen, deren euklidische Distanz zur aktuell betrachteten Pose geringer ist als eine festgelegte Schwelle, werden als Kandidaten angesehen. Jeder dieser Kandidaten werde über ein Scan-Matching der zugehörigen Punktwolken und der Evaluation des Ergebnisses des Scan-Matching validiert \cite{shan2020lio}. \cite{shan2020lio} verwendet einen Faktor-Graphen zur Speicherung der Pose-Relationen. Hierfür wird die Implementation des Faktor-Graphen von \emph{GTSAM} \cite{dellaert2012factor} verwendet, der ebenfalls in dieser Arbeit zur Speicherung und Optimierung der Pose-Relationen verwendet wird.
Wichtige Voraussetzung für die Optimierung der Posen im Graphen ist eine Assoziation der Posen mit den zugehörigen Umgebungsdaten. Im Falle von Borrmann et al. \cite{borrmann2008globally} sind dies jeweils die zum Zeitpunkt aufgenommenen Punktwolken. Diese Assoziation ist bei TSDF-basiertem SLAM, das eine diskretisierte, volumetrische TSDF-Karte zur Speicherung von Umgebungsdaten verwendet, nicht explizit möglich, solange nicht zusätzlich die entsprechenden Punktdaten abgespeichert werden. Wird ein Schleifenschluss detektiert, muss die gesamte Karte auf Basis der abgespeicherten Punktdaten neu erstellt werden. Dies ist speicher- und zeitaufwändig. Dieses Problem markiert den Hauptbeitrag dieser Arbeit. Es gilt zu untersuchen, welcher Teil der TSDF-Karte mit welcher Pose im Graph assoziiert werden kann, um anschließend die bereits existierende Karte zu modifizieren und zu optimieren. Dieses Problem wird im Folgenden untersucht und verschiedene Herangehensweisen werden eruiert.


\section{Motivation}

Ziel dieser Arbeit ist die Lösung des Schleifenschluss Problems für TSDF basierte SLAM Verfahren.
Als Basis dient der Ansatz von Eislodt et al. \cite{HATSDF}, der -- obgleich performant und funktional -- wie die meisten SLAM Verfahren bei der Kartierung großer Umgebungen stark anfällig für eine Akkumulation von Fehlern ist, was die Kartierung großer Areal deutlich erschwert. Bislang sind in \citep{HATSDF} keine Algorithmen implementiert, die in der Lage sind diese akkumulierten Fehler zu kompensieren. Um dies zu erreichen, wird in dieser Arbeit die Detektion von Schleifenschlüssen mit anschließender Pose-Graph- und Kartenoptimierung angestrebt. Zudem erfolgt eine Evaluation der an dem so entwickelten Ansatz beteiligten Operationen in Hinsicht auf deren Robustheit und Verwendbarkeit. Ein wesentlicher Schritt ist hier die Optimierung der TSDF-Karte basierend auf einem zuvor optimierten Pose-Graphen, da im Gegensatz zu Verfahren, die Punktdaten als Repräsentation der Karte verwenden, keine Assoziation zwischen den Zellen der TSDF-Karte und den Posen des Graphen ablesbar sind. Ziel ist die Untersuchung dieser Gegebenheit und die Entwicklung von Lösungsstrategien für dieses Problem. Diese Arbeit soll als \emph{Proof-of-Concept} für weitere Arbeiten auf diesem Gebiet dienen und Herausforderungen und Fallstricke aufzeigen. Es soll geklärt werden wie Schleifenschlüsse in einen TSDF-basierten SLAM-Ansatz integriert werden können, ob eine Nachbehandlung auf Basis einer initialen Schätzung möglich ist und unter welchen Voraussetzungen ein partielles Update der TSDF-Karte durchgeführt werden kann.

\section{Vorgehensweise}
\label{section:herangehensweise}

Das Ergebnis der im vorigen Abschnitt erwähnten Untersuchung ist offen. Je nach Ergebnis ist in dieser Arbeit entsprechend zu reagieren und die weitere Vorgehensweise offen darzulegen. In einem ersten Ansatz wird dazu untersucht, ob die fehlende Relation zwischen dem Pose-Graphen und der TSDF-Karte durch die Identifikation von Datenassoziationen zwischen den beiden Strukturen kompensiert werden kann und die so generierten Daten für ein Update der TSDF-Karte verwendet werden können. Auf Basis dieser Untersuchung soll eine Einschätzung zur Nutzung von Schleifenschlüssen in einem TSDF basierten SLAM Ansatz als Nachbehandlungs-/\emph{Post-Processing}-Schritt geben werden. Diese Einschätzung markiert einen wesentlichen Meilenstein dieser Arbeit. Es gilt zu entscheiden, ob sich das Szenario der Nachbehandlung als möglich erweist und weiter verfolgt wird, oder  stattdessen eine Integration in den zugrunde liegenden SLAM-Ansatz angestrebt wird. Dies erfordert in vielerlei Hinsicht ein Umdenken, da auf einer größeren Datenbasis operiert werden kann, die zusätzlich die von einem Laserscanner aufgenommenen Punktwolken enthält. Ziel ist hier die Algorithmik zu konzipieren und entwickeln, die zu einem gegebenen Zeitpunkt untersucht, ob eine Graph-Optimierung auf Basis der aktuellen Umgebungsdaten möglich ist und die Schritte von der Detektion über die Optimierung des Pose-Graphen bis hin zur Optimierung der Karte realisiert. Um diese Einschätzung treffen zu können liegt der Fokus des folgenden Kapitels auf der Generation relationaler Daten zwischen dem Pose-Graphen und der TSDF-Karte. Es folgt eine Einschätzung der so generierten Daten mit besonderer Rücksicht auf das zu lösende Schleifenschluss-Problem. Es wird eine evaluative Schätzung abgegeben, ob die Lösung des Problems in der Nachbehandlung möglich ist und dabei ermittelte Probleme, Fallstricke und Erkenntnisse werden herausgestellt. Basierend auf den Ergebnissen der Evaluation wir die weitere Vorgehensweise am Ende des folgenden Kapitels diskutiert. In einem Ausblick wird eine mögliche Beschleunigung der vorgenommenen Implementationen diskutiert, mit dem Ziel, sie zukünftig in ein Echtzeit-Kartierungssystem wie HATSDF-Slam integrieren zu können. Die Implementation dieses Prototyps wird ausschließlich in Software realisiert, allerdings erfolgt eine Untersuchung des Software-Prototyps auf Potenziale zur Hardware-Beschleunigung um Raum für Optimierungen im Rahmen zukünftiger Arbeiten zu eröffnen.


