\chapter{Stand der Forschung}\label{chapter:sdf}

Dieses Kapitel untersucht den aktuellen Stand der Forschung und gibt einen Überblick über das Forschungsgebiet. Dazu wird zunächst im Grundsatz auf das Thema \emph{SLAM}, sowie auf das Thema Schleifenschlüsse eingegangen. Im Anschluss erfolgt eine Spezifizierung auf TSDF basierte SLAM-Verfahren, sowie die Ursprünge der TSDF. Der Fokus liegt hier auf der TSDF und den Forschungsbereich der TSDF-basierten Verfahren, die in den kontextuellen Rahmen dieser Arbeit passen. Diese Verfahren sind im Abschnitt \emph{TSDF} dargelegt.

\textbf{SLAM}

Dieser Abschnitt stellt im Groben den Stand der Forschung des SLAM-Problems heraus
\textit{Simulataneous Localization and Mapping (SLAM)} ist eines der größten Forschungsgebiete in der mobilen Robotik \cite{cadena2016past} und beschreibt den Prozess der Lokalisierung in einer Karte auf Basis ankommender Sensordaten bei einer gleichzeitigen Generation dieser Karte. Die Forschungen zu diesem Problem reichen zurück bis vor die Jahrtausendwende. SLAM besteht aus einer Lokalisations- und einer Kartierungskomponente, die aufeinander aufbauen, aber getrennt voneinander betrachtet werden können. Verschiedene Lokalisierungsansätze auf Basis von Scan-Matching-Ansätzen wie \emph{Iterative Closest Point (ICP)} nach Besl \& McKay \cite{besl1992method}, \emph{Normal ICP (NICP)} \cite{chen1992object} oder \emph{Generalized ICP (GICP)} \cite{segal2009generalized} werden in der Praxis verwendet (\cite{shan2020lio, izadi2011kinectfusion}). Diese Algorithmen benötigen dabei in der Regel eine gute initiale Pose-Schätzung \cite{he2017iterative}. Andere Ansätze nutzen statistische Modelle zur Lokalisierung. So verwenden Montemerlo \cite{montemerlo2002fastslam, montemerlo2003fastslam} einen Partikelfilter um statistisch abschätzen zu können, welche Pose ein Roboter oder Sensorsystem derzeit annimmt. Im Bereich der Kartierung wird in der Regel zwischen expliziten und impliziten Darstellungen der Oberfläche abgetasteter Objekte unterschieden. Explizite Darstellungen können zum Beispiel aus einer Ansammlung inkrementell registrierter Punktwolken bestehen \cite{izadi2011kinectfusion}, wobei in vielen dieser Ansätze nicht jede Punktwolke, sondern nur sogenannte \emph{Key-Frames} genutzt werden \cite{shan2020lio}. Eine Punktwolke gilt als Key-Frame, wenn die Distanz die zugehörigen Pose zur Pose des letzten Key-Frames größer ist, als eine vordefinierte Distanz \cite{shan2020lio}.
Wesentlicher Faktor für die Robustheit von SLAM Ansätzen ist die Integration von Schleifenschlüssen. Im Folgenden wird ein grober Überblick über dieses Thema geliefert. Implizite Darstellungen beschreiben den Raum um die abgetastete Oberfläche herum und ermöglichen eine Kontinuität, die durch eine Beschreibung der Oberfläche mit Punktdaten nicht erreicht werden kann \cite{curless1996volumetric}. Dazu gehören zum Beispiel vorzeichenbehaftete Distanzfelder oder Distanzfunktionen, deren Ursprung und Forschungsstand in Kapitel \emph{TSDF} näher behandelt wird. 

Die Forschung im Bereich SLAM hat sich in verschiedene Richtungen entwickelt, die auf unterschiedlichen Grundlagen oder Annahmen über vorhandene oder ermittelte Daten arbeiten. Dissanayake et al. \cite{dissanayake2001solution} schlagen einen SLAM Ansatz basierend auf Landmarken und einem \emph{Erweiterten Kalman Filter (EKF)} vor. Andere Ansätze extrahieren Features aus Punktwolken und nutzen diese für einen anschließenden Datenvergleich zur Bestimmung der vorliegenden Pose-Differenz. Dazu gehören zum Beispiel zum Beispiel die Arbeiten von Zhang \& Singh \cite{zhang2014loam} oder Shang \& Englot \cite{shan2018lego}. Wiederum andere Ansätze nutzen eine gesonderte Betrachtung einer \emph{Inertial Measurement Unit (IMU)}, um eine optimierte initiale Schätzung der vorliegenden Pose-Differenz zu erreichen. Um sich innerhalb dieser Forschungsgebiete zurechtzufinden, existieren diverse Arbeiten, die eine umfangreiche und detaillierte Beschreibung verschiedener SLAM-Ansätze angefertigt haben \cite{cadena2016past, aulinas2008slam, taketomi2017visual}. 


\textbf{Schleifenschlüsse}

Um genannte SLAM-Ansätze robust gegenüber einer Akkumulation von Fehlern zu machen, die häufig in dieser Kategorie von Algorithmen auftreten, werden Schleifenschlüsse verwendet, anhand derer die Fehler nachträglich korrigiert werden können. Grundidee ist dabei anhand der gerade aufgenommenen Sensordaten zu beurteilen, ob diese zu bereits im Vorfeld aufgenommenen Daten passt und darauf basierend die erstellte Karte oder den generierten Pose-Graphen zu optimieren. Für diese Einschätzung existieren mehrere Varianten unterschiedlicher Komplexität, jeweils basierend auf der Bestimmung von Kandidaten und deren anschließenden Verifikation. Im Folgenden wird ein Überblick über das Thema der Wiedererkennung von Orten beziehungsweise Schleifenschlüsse gegeben, welches in der Literatur häufig als \emph{Loop-Closure}, \emph{Place-Recognition} oder \emph{Revisiting-Problem} bezeichnet wird, wobei die genannten Bezeichnungen nicht dasselbe Thema bezeichnen, sondern lediglich eng miteinander verwandt sind. Garg et al. \cite{garg2021your} liefern einen breiten Überblick über die Unterschiede zwischen der visuellen \emph{Place-Recognition} in der Robotik und der Computergrafik. Der im Folgenden dargelegte Überblick über Schleifenschlüsse in der Robotik deckt bei Weitem nicht den Umfang der Forschungen in diesem Bereich ab. 

Frühe Forschungen in diesem Bereich stammen von Lu \& Milios \cite{lu1997globally}, die einen Ansatz zur konsistenten, globalen Registrierung von 2-D Reichweiten-Scans vorschlagen. Dazu nutzen sie ein Netzwerks aus relativen Pose-Beschränkungen, die durch die Implementation eines Scan-Matching Ansatzes und Odometrie ermittelt werden, welches mit einem globalen Ansatz optimiert wird. Zu diesen Beschränkungen zählen auch ermittelte Schleifenschlüssen. Ho \& Newman \cite{ho2006loop} stellen heraus, dass bei herkömmlichen Verfahren zur Identifikation von Schleifenschlüssen eine Initialschätzung notwendig ist und schlechte Schätzungen zu fehlerhaften Schleifenschlüssen führen können. Sie schlagen stattdessen ein Verfahren vor, welches unabhängig von der Navigation und einer damit verbundenen initialen Poseschätzung ist und sich auf den Abgleich charakteristischer Signaturen einzelner lokale Szenen beziehungsweise Scans zur Detektion von Schleifenschlüssen stützt. Borrmann et al. \cite{borrmann2008globally} schufen in ihrer Arbeit eine gute Grundlage für die Integration von Schleifenschlüssen in SLAM-Verfahren auf Basis von Pose-Graphen, die bis heute vielfach verwendet wird. Darauf aufbauend optimieren Sprickerhof et al. \cite{sprickerhof2011heuristic} den Schleifenschluss-Ansatz durch Nutzung einer heuristischen Schleifenschluss-Technik, die im Gegensatz zu bisherigen Methoden einen dünn besetzten SLAM-Graphen nutzen und vergleichbare Ergebnisse bei einer reduzierten Laufzeit erzielen. In \cite{lowry2015visual} stellen eine umfangreiche Forschung zur visuellen \emph{Place-Recognition} im Robotik-Kontext dar. Xie et al. \cite{xie2017graphtinker}, sowie Tsintotas et al. \cite{tsintotas2022revisiting} stellen die Notwendigkeit von weiteren Validierungsstufen für Schleifenschlüsse heraus um die Anzahl fehlerhafter Schleifenschlüsse und deren verhängnisvolle Auswirkungen auf die Optimierung eines Pose-Graphen zu reduzieren. Ähnlich wie \cite{borrmann2008globally} nutzen Shan et al. \cite{shan2020lio} in ihrem Feature-basierten SLAM Ansatz die euklidische Distanz zur Bestimmung von Kandidaten für Schleifenschlüsse, die durch eine Evaluation des \emph{Fitness-Scores} von ICP nach Registrierung der zugehörigen Punktwolken der beiden Posen der jeweiligen Kandidaten verifiziert werden. Zur Optimierung des Pose-Graphen, basierend auf den so identifizierten Schleifenschlüssen, verwendet \cite{shan2020lio} die Bibliothek GTSAM \cite{dellaert2012factor} und deren Implementation eines Faktor-Graphen. GTSAM kapselt dabei die Funktionalität der Graph-Optimierung und ermöglicht so eine vereinfachte Optimierung basierend auf detektierten Schleifenschlüssen, die vielfach genutzt wird (\cite{shan2020lio}). Eine detaillierte Übersicht über existierende Verfahren zur Detektion von Schleifenschlüssen mit Bezug zu früheren Forschungen auf diesem Gebiet ist in dem Papier von Tsintotas et al. \cite{tsintotas2022revisiting} dargelegt.  

Im Folgenden wird eine Übersicht des Forschungsstands im Bereich der TSDF dargelegt mit besonderem Bezug auf TSDF-basierte SLAM Verfahren, die im vorigen Abschnitt nur grob angeschnitten wurden.

\textbf{TSDF}

Die in dieser Arbeit entwickelten Algorithmen und Konzepte basieren oder zielen auf eine implizite, volumetrische Repräsentation der Umgebung eines Roboters oder Sensorsystems, die \emph{Truncated Signed Distance Function (TSDF)}. Erste Forschungen in diesem Themengebiet stammen aus der Computergrafik, genauer der Rekonstruktion von Oberflächen aus räumlichen Daten. Hoppe et al. definieren und demonstrieren einen Algorithmus, der aus einer gegebenen Menge von unorganisierten Punkten $\left\lbrace x_1, ..., x_n \right\rbrace \in \mathbb{R}^3$ eine vereinfachte Oberfläche generiert. Dazu wird eine vorzeichenbehaftete geometrische Distanzfunktion verwendete, die Distanz zur unbekannten Oberfläche abschätzt. Ein solche Distanzfunktion wird ebenfalls von Bajaj et al. \cite{bajaj1995automatic} die einen Ansatz zur Rekonstruktion der Oberfläche von \emph{(Computer Aided Design (CAD)} Modellen und deren skalaren Feldern auf Basis einer unorganisierten Sammlung von gescannten Punktdaten. Basierend auf diesen Forschungen stellen Curless \& Levoy \cite{curless1996volumetric} eine volumetrische Methode zur Konstruktion komplexer Modelle auf Basis einer Reihe von \emph{Range-Images}, die zum Beispiel durch Tiefenbildkameras oder \emph{Light Detection and Ranging(Lidar)}-Sensoren generiert werden, vor. Sie basieren die Rekonstruktion der Oberfläche auf einer vorzeichenbehafteten Distanzfunktion \emph{SDF}. Der vorgestellte Ansatz ermöglicht ein inkrementelles Update, die Repräsentation gerichteter Ungenauigkeit, sowie das Füllen von Lücken beziehungsweise Löchern in der SDF und ist robust gegen \emph{Outlier} \cite{curless1996volumetric}. Zusätzlich ist der vorgestellte Ansatz unabhängig von der Reihenfolge, in der die Scans in die SDF eingefügt werden, wodurch kein Bias durch zuerst eingefügte Scans entsteht. Dieser Ansatz stellt die Basis für viele weitere Ansätze, die im Bereich der Rekonstruktion im Bereich der SDFs operieren. So präsentieren Breen et al. \cite{breen19983d} einen Ansatz zur Generation eines Distanz-Volumens auf Basis von \emph{Constructive Solid Geometry (CSG)} Modellen. Um die Probleme der zuvor genannten Ansätze mit Objekten veränderlichen Detailgrades zu lösen, schlagen Frisken et al. \cite{frisken2000adaptively} eine adaptives, vorzeichenbehaftetes Distanzfeld \emph{(ADF)} vor, welches als eine generelle Repräsentation für Formen in der Computer-Grafik dienen soll, vor. Diese ist nach \cite{frisken2000adaptively} vorteilhaft für räumliche Strukturen mit Features in sehr unterschiedlichen Maßstäben. In ADFs werden Abstandsfelder entsprechend der lokalen Details adaptiv abgetastet und in einer räumlichen Hierarchie für eine effiziente Verarbeitung gespeichert. Bastos \& Celes optimieren den Ansatz aus \cite{frisken2000adaptively} durch die Nutzung eines \emph{Grafikprozessors (GPU)}. Um scharfe Details in TSDF repräsentieren zu können schlagen Novotny \& Sramek \cite{novotny2005representation} einen Ansatz vor, der die durch das TSDF repräsentierten Objekte nach dem Kriterium der Darstellbarkeit zu verändern. Zusätzlich werden Informationen über die Oberflächen-Normalen mit berücksichtigt.
Jones et al. \cite{jones2006distance} stellen eine Übersicht über 3-D Distanzfelder vor und gehen dabei auf die verwendeten Techniken und Anwendungsbereiche ein. Sie stellen dabei unter anderem heraus, dass Methoden, die Distanzfelder nur bis zum einem festgelegten Maximalwert erzeugen, also die Charakteristiken einer TSDF aufweisen, in den Ausführungen von Mauch \cite{mauch200fast, mauch2003efficient} ihren Ursprung haben. Weitere Forschungen in diesem Themengebiet befassen sich zum Beispiel mit der Parallelisierung der volumetrischen Repräsentation. Werner et al. \cite{werner2014truncated} evaluieren die Ergebnisse der Oberflächenrekonstruktion anhand von TSDF-Karten unter Berücksichtigung der Größe der Voxel und geben einige Empfehlungen für den Umgang mit TSDF.

Erste Forschungen mit TSDF basiertem SLAM stammen aus dem letzten Jahrzehnt.
Izadi et al. \cite{izadi2011kinectfusion} nutzen ein TSDF-Voxelgrid und eine Kinect-Tiefenkamera, um Umbegungen zu kartieren, während ein Nutzer die Kinect Kamera durch die Umbebung schwenkt. Dabei wird das aktuell betrachtete Tiefenbild der Kinect-Kamera jeweils in eine Punktwolke konvertiert und mittels ICP an die letzten Tiefenbilder registriert. Ergebnis der Registrierung ist eine 6-D Pose für die aktuelle Position der Kamera. Auf Basis der so erhaltenen Kameraposition erfolgt im Anschluss ein Update des volumetrischen Oberflächenrepräsentation, die ebenfalls auf den Ausführungen aus \cite{curless1996volumetric} basiert. Whelan et al. \cite{whelan2012kintinuous} optimieren diesen Ansatz durch Nutzung eines Ringbuffers zur Kartierung großer Umgebungen. Im Gegensatz zu diesen Verfahren stellt Canelhas \cite{Canelhas2017TruncatedSD} ein neues Verfahren zur Registrierung neuer Punktdaten an die TSDF-Karte vor: \emph{Point-to-TSDF}. Dabei werden die Punktdaten mittels eines Gradientenabstiegs an das TSDF registriert. Ergebnis ist erneut eine 6-D-Pose, anhand derer das TSDF erweitert werden kann. Zusätzlich liefert Canelhas \cite{Canelhas2017TruncatedSD} einen Überblick über das Themengebiet der \emph{TSDF}. Prisacariu et al. \cite{prisacariu2017infinitam} unterteilen den 3D-Raum in starre TSDF-Teilkarten und optimieren die relativen Positionen zwischen diesen. Nach \cite{prisacariu2017infinitam} ist eine anschließende Generation der globalen Karte durch ein Zusammenführen der Teilkarten möglich. \cite{prisacariu2017infinitam} verwendet für jede Pose des Graphen eine eigene Teilkarte, die in sich konsistent ist. Sie erreichen eine globale Konsistenz durch eine Graph-Optimierung, die zusätzlich Schleifenschlüsse berücksichtigt. Optimiert werden die zu den Teilkarten zugehörigen Posen. Die TSDF-Teilkarten selbst werden durch die Graph-Optimierung nicht angepasst. In einem weiteren volumetrischen Ansatz von McCormac et al. \cite{mccormac2018fusion++} wird eine volumetrischer, Objekt basierter SLAM Ansatz verwendet, der eine TSDF zur Repräsentation mehrerer Objekte verwendet. Dabei werden lediglich die relativen Pose-Schätzungen zwischen den Objekten verändert, aber keine Verformung der Objekte selbst durchgeführt. Eisoldt et al. \cite{HATSDF} schlagen einen mit FPGA Hardware beschleunigten, TSDF-basierten SLAM Ansatz vor und verwenden einen 3-D-Laserscanner anstelle einer Tiefenkamera. Sie nutzen zur Registrierung neuer Punktwolken an die TSDF-Karte den von Canelhas \cite{Canelhas2017TruncatedSD} vorgeschlagenen \emph{Point-to-TSDF} Ansatz, der die Distanzen von Punkten einer neuen Punktwolke zur durch die TSDF implizit beschriebenen Oberfläche entlang des Gradienten innerhalb des TSDF minimiert. \cite{HATSDF} implementiert keinen Ansatz zur Integration von Schleifenschlüssen und Graphoptimierung und ist dementsprechend anfällig für akkumulative Fehler. An dieser Stelle setzt diese Arbeit nun an und integriert auf Basis der zuvor genannten Ansätze, Schleifenschlüsse in einen TSDF basierten SLAM Ansatz und die TSDF Karte entsprechend der Änderungen an der Roboter-Trajektorie zu optimieren.





