\chapter{Stand der Forschung}\label{label:sdf}



\textit{Simulataneous Localization and Mapping (SLAM)} ist eines der größten Forschungsgebiete in der mobilen Robotik.
Die Forschungen zu diesem Problem reichen zurück bis vor die Jahrtausendwende.
Einer der bekanntesten Lösungen zur Registierung dreidimensionaler Daten ist der ICP Algorithmus nach Besl \& McKay \cite{besl1992method}, der in einer Abwandlung ebenfalls von Borrmann et al. im GraphSLAM \cite{borrmann2008globally} verwendet wird.
Bis heute gibt es zahlreiche Implementationen dieses Algorithmus.

Erste Forschungen mit TSDF basiertem SLAM stammen aus dem letzten Jahrzehnt.
Izadi et al. \cite{izadi2011kinectfusion} nutzen ein TSDF-Voxelgrid und eine Kinect-Tiefenkamera, um Umbegungen zu kartieren, während ein Nutzer die Kinect Kamera durch die Umbebung schwenkt.
Whelan et al. \cite{whelan2012kintinuous} optimieren diesen Ansatz durch Nutzung eines Ringbuffers zur Kartierung großer Umgebungen.
Im Gegensatz zu genannten TSDF Verfahren, nutzen Eisoldt et al. \cite{HATSDF} einen Hardware beschleunigten TSDF basierten SLAM Ansatz, sowie einen 3D-Laserscanner anstelle einer Tiefenkamera.

Borrmann et al. \citep{borrmann2008globally} schufen in ihrer Arbeit eine gute Grundlage für die Integration von Schleifenschlüssen in SLAM-Verfahren auf Basis von Pose-Graphen, die bis heute vielfach verwendet wird.

\improvement{Insert more tsdf based approaches from \cite{werner2014truncated}}

Ziel und Zweck dieser Arbeit ist die Lösung des Schleifenschluss Problems für TSDF basierte SLAM Verfahren.
Als Basis dient der Ansatz von Eislodt et al. \cite{HATSDF}, der - obgleich performant und funktional - wie die meisten SLAM Verfahren bei der Kartierung großer Umgebungen stark anfällig für Drift ist. Um diesem Problem entgegen zu wirken, soll durch die Integration von Schleifenschlüssen in TSDF-Karten die Kartierung größerer Umgebungen ermöglicht werden.
Um dieses Ziel zu erreichen, gilt es Datenassoziationen zwischen Posen und Zellen in der diskretisierten TSDF Karte herzustellen.
In einer ersten Implementation soll dazu mittels Ray-Tracing eines simulierten Laserscans und der Detektion von Schnittpunkten mit der TSDF-Karte eine passende Indizierung der Zellen ermöglicht werden.

Diese Arbeit soll als Proof-of-concept für weitere Arbeiten auf diesem Gebiet dienen. Dazu werden die entwickelten Ansätze in einem Post-Processing Schritt auf die vom Eisoldt et al. \cite{HATSDF} erstellte Karte, so wie den Pfad, den es noch zu extrahieren gilt, angewandt.


In einem Ausblick gilt es zu analysieren, ob die Implementationen sinnvoll beschleunigt werden können, um sie in ein Live Karierungssystem, wie HATSDF-Slam einbauen zu können, um zur Laufzeit Schleifen zu erkennen und die Karte zu optimieren.

Die Implementation dieses Prototyps wird ausschließlich in Software realisiert, allerdings erfolgt eine Untersuchung des Software-Prototyps auf Potenziale zur Hardware-Beschleunigung um Raum für Optimierungen im Rahmen zukünftiger Arbeiten zu eröffnen.

TODO: Update der Motivation? 
Untersuchung, ob das Loop Closing als Post-Processing angewandt werden kann, gegebenenfalls Integration in inkrementellen SLAM
Möglichkeit schaffen mit beliebigen SLAM Verfahren koexistieren zu können (LC)
Schaffen einer API?
Untersuchen welche Art des Kartenupdates sinnvoll ist und welches nicht
Untersuchen, ob der Aufwand die Karte im Teil zu updaten gerechtfertigt ist, Untersuchung von Fallstricken

\section{Ansatz}

Beschreiben, wie der initiale Aufbau dieser Arbeit ist:

1. Herangehensweise, wie soll das Problem gelöst werden?
2. Offenheit des Themas klar herausstellen
3. Mögliche Fallback's aufzeigen, falls sich herausstellt, dass das Thema so in der Bearbeitungszeit nicht möglich ist
4. Deutlich machen, dass die erste Herangehensweise dokumentiert wird
5. Erklären, dass Fallstricke und Probleme erläutert werden
6. Lösungsansätze für zukünftige Arbeiten definieren
7. Kenntlich machen, warum das Problem "inkrementelles Map Update" nicht in der Zeit lösbar gewesen wäre

\section{Fallback}


