\chapter{Loop Closure}
\label{chapter:loop_closure}

Diese Sektion baut auf den Grundlagen aus Sektion \ref{section:loop_closure_basics} auf.

\section{Detektion}

Beschreibung der Detektion (bildhaft), Erklärung von Kandidaten und Filterung
Beschreibung des optionalen Sichtbarkeitskriteriums

\section{Graphenoptimierung}

Bezug zu GTSAM Library Beschreibung aufnehmen.
Wichtigste verwendete Funktionen bennen.
Verweis auf GTSAM Paper.
Erklärung von Faktorgraphen und Faktoren.
Erklärung der Genutzten datenstrukturen und optimizer.
Erklärung von noise constraints (Unsicherheiten)

\section{Optimierungen}

1. Vorregistrierung

2. Filtern der Punktwolke

3. Unterschiedliche Scan-Matching Varianten
	1. ICP
	2. GICP
	3. Kurz auf Teaser++ eingehen
	4. VGICP
	
	-> Analyse des Scan Matchings bezogen auf 1. Vorregistrierung, 2. LC Detektion Matching
	
\section{Datensätze}

Verwendete Datensätze und herausforderungen herausstellen

\subsection{Hannover1}

Herausforderungen:

- Datensatz allgemein: falsches Koordindatensystem -> Transformation beschreiben

- extrem fehlerbehaftete Rotation in zweitem Kreis (Lösung: Vorregistrierung ICP)
(Scan-Matching bekommt extrem divergierte Wolken nicht mehr aufeinander)

- problematisch: fehlerhafte LC Optimierung durch schlechtes Scan Matching (Grund: Feature-armer Flur) -> mögliche ( noch zu entwickelnde) Lösung: LC's auf Linien gesondert betrachten -> Identifikation eines LC auf Linien -> Betrachtung der Scan Matching Transformation - wenn aufGeraden kann die Transformation die beiden LC-Punkte nicht vollständig zusammen ziehen

\subsection{Maps}

